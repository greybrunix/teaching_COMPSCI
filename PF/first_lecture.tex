%\documentclass[compress,svgnames,13.7pt]{beamer}
\documentclass[compress,svgnames,handout,13.7pt]{beamer}
\usepackage[english]{babel}
\usepackage{makeidx,verbatim}
\usepackage{latexsym,amsfonts,amsmath,amssymb,amsthm}
\usepackage{curves}
\usepackage{emerald}
\usepackage{enumerate}
\usepackage{moreverb}
\usepackage{cases}
\usepackage{array}
\usepackage{epsfig}
\usepackage{graphics}
\usepackage{float}
\usepackage{color}
\usepackage{stmaryrd}
\usepackage{amstext}
\usepackage{xspace}
\usepackage[mathcal]{euscript}
\usepackage{rotating}
\usepackage{mathrsfs}
\usepackage{pgf,pgfnodes,pgfheaps}
\usepackage{tikz}
\usepackage{listings}
\usepackage{xcolor}
\usetikzlibrary{arrows,automata,backgrounds}
\usecolortheme{seahorse}
\usecolortheme{rose}
\usefonttheme[onlylarge]{structuresmallcapsserif}
\setbeamerfont{frametitle}{family=\rmfamily,size=\footnotesize}
\setbeamercolor{title}{fg=blue!80!black}%,bg=blue!20!white}
\setbeamercolor{frametitle}{fg=blue!80!black,bg=blue!20!white}
\setbeamertemplate{miniframes}[tick]
\setbeamertemplate{blocks}[rounded][shadow]
\setbeamerfont{small}{size=\small} \setbeamerfont{tiny}{size=\tiny}
\setbeamercolor{caixa}{fg=black,bg=blue!10!white}

\def\N{{\mathbb N}}
\newcommand{\Z}{\mathbb Z}
\newcommand{\R}{\mathbb R}
\newcommand{\Q}{\mathbb Q}
\def\dotminus{\mathbin{\ooalign{\hss\raise1ex\hbox{.}\hss\cr\mathsurround=0pt$-$}}}
\def\proof{\noindent{\bf\blue Proof}\ }

\def\red{\color[rgb]{0.8,0,0}}
\def\lightred{\color[rgb]{1,0.5,0.5}}
\def\green{\color[rgb]{0,0.5,0}}
\def\lightgreen{\color[rgb]{0.5,1,0.5}}
\def\blue{\color[rgb]{0,0,0.7}}
\def\darkredblue{\color[rgb]{0.4,0,0.4}}
\def\lightgray{\color[rgb]{0.7,0.7,0.7}}
\def\mystep#1#2{\uncover<#1->{{\color<#1>[rgb]{0,0,1}{#2}}}}
\definecolor{azul}{rgb}{0,0,.7}
\definecolor{codegreen}{rgb}{0,0.6,0}
\definecolor{codegray}{rgb}{0.5,0.5,0.5}
\definecolor{codepurple}{rgb}{0.58,0,0.82}
\definecolor{backcolour}{rgb}{0.95,0.95,0.92}
\tikzset{->,
    >=stealth',
    node distance=3cm,
    every state/.style={thick, fill=gray!10},
    initial text=$ $,
}

\lstdefinestyle{mystyle}{backgroundcolor=\color{backcolour},
    commentstyle=\color{codegreen},
    keywordstyle=\color{magenta},
    numberstyle=\tiny\color{codegray},
    stringstyle=\color{codepurple},
    basicstyle=\ttfamily\footnotesize,
    breakatwhitespace=false,
    breaklines=true,
    captionpos=b,
    keepspaces=true,
    numbers=left,
    numbersep=5pt,
    showspaces=false,
    showstringspaces=false,
    showtabs=false,
    tabsize=2
}

\lstset{style=mystyle}

\renewcommand\qedsymbol{QED}

\def\GHC{Glasgow Haskell Compiler\xspace}

\title[Functional Programming]
    {\textbf{Functional Programming}\\
    Using Haskell
}

\author[Bruno G.]
   {Bruno Dias da Gião
}

\date{\today}

\usetheme{CambridgeUS}
%\setbeamercovered{transparent}
\setbeamercovered{invisible}


\AtBeginEnvironment{tikzpicture}{\tracinglostchars=0\relax}
\begin{document}


% Title Page
\thispagestyle{empty}
\frame{\titlepage}


\begin{frame}{Contents}
\tableofcontents
\end{frame}


\section{Introduction}
\begin{frame}
    \Huge{\textbf{Introduction}}
\end{frame}
% SLIDE 0
\subsection{Introductory Questions}
\begin{frame}
    \begin{enumerate}
        \item{What is Functional Programming?}
    \pause\item{Why is it useful?}
    \pause\item{When do we get to C or Python or Java?}
    \end{enumerate}
\end{frame}
% SLIDE 1
\subsection{Installing Haskell via the \GHC}
\begin{frame}
    \begin{enumerate}
        \item{Installing chocolatey} You will need to come to this link:
            https://chocolatey.org/install\#individual and follow the
            instructions for ``Individual use''
        \item{Installing cabal and ghc} Run Powershell in admin mode and
            run the following commands \[\text{choco install ghc}\]
            \[\text{choco install cabal}\]
    \end{enumerate}
\end{frame}

\section{Functions}
\begin{frame}
    \Huge{\textbf{Functions}}
\end{frame}
\subsection{What are function?}
\begin{frame}
    \huge{We still don't know what exactly is a function\ldots}
    \par
    \large{In that case let's try to define it in a very broad way\ldots as a contract
        \par What does that mean? It means a function is something that receives something
    and gives us something}\par
    But, what are those\ldots ``Somethings''?
\end{frame}
\subsection{Functions as a Contract}
\begin{frame}
    \begin{enumerate}
    \item{Input/Arguments to a Function}
    \pause\begin{definition}[Input]
        An input is something that we pass to the function.
    \end{definition} 
        For example, ingredients are the input of the function that bakes a cake.
    \pause\item{Output of a Function}
    \pause\begin{definition}[Output]
        The output of a function is what the function gives us back.
        \end{definition}
        In the cake example, the cake itself is the output!
    \end{enumerate}
\end{frame}

\begin{frame}
    \huge{\textbf{What exactly binds the definitions of Input and Output?}}
    \par
    \pause\large{\textbf{Before we get to that let's do some exercises}}
\end{frame}
\subsection{Types}
\begin{frame}
    \begin{definition}[Type]
        A type is a Set whose's elements can be inserted into a function
        or be received from a function. Every function has a set of types
        for input and output.
    \end{definition}
    \pause\begin{example}
        In the function ``bake a cake'', we would grab a set of ingredients,
        perform a series of instructions, and then we would receive one of the many
        cakes that can be made. \hfill \par
        \pause However, in this very same function, we cannot give it a cake,
        and expect anything, or give it ingredients and expect ingredients,
        the types don't match those of the function!
    \end{example}
\end{frame}

\begin{frame}
    \Huge{But what \textbf{are} those series of instruction?}
\end{frame}
\subsection{From Input to Output}

\begin{frame}
    During the exercises we managed to find 
    an ``effect'' that removed unnecessary spaces from
    sentences and preserved the needed spaces. \pause\par
    \Huge{That \textbf{was} a function.}
    \pause\normalsize{}\begin{definition}[Function]
    A function is the procedure that transform an
    input into an output.
    \end{definition}
    \pause\begin{definition}[Defining a function]
    Thus we can define a function as such:
    \[\text{name\_of\_function input = output}\]
    Where the output is achieved via \textbf{Composition}
  \end{definition}
\end{frame}
\subsection{Inside the Black Box}
\begin{frame}
    \begin{definition}[Composition]
        Let f and g be two functions. We say they are composed
        when the output of one of them is fed into the other as
        input.
        \pause\begin{equation}
            f \circ g = f(g(x))
        \end{equation}
        In other words, $g$ receives $x$ as an input and $f$ receives
        the result of $g(x)$ as the input.
    \end{definition}
    \pause\begin{corollary}
        The composition of two functions (if correctly defined), is
        itself a function.
    \end{corollary}
\end{frame}

\begin{frame}
    \begin{example}
        One of the exercises we did involved on such composition:
        \[\text{unwords (words sentence)}\] is an example of the composition of
        functions.
    \end{example}
\end{frame}

\end{document}
