% Template for taking notes\writing down exercise solutions
% In the specific context of mathematical Curricular Units

% In order to draw Finite State Machines\Turing Machines,
% we make use of the tikz-automata library

\documentclass[a4paper,10pt]{article}

\usepackage{xspace}
\usepackage[english]{babel}
\usepackage{amsmath}
\usepackage{amsthm}
\usepackage{amssymb}
\usepackage{listings}
\usepackage{xcolor}

\definecolor{codegreen}{rgb}{0,0.6,0}
\definecolor{codegray}{rgb}{0.5,0.5,0.5}
\definecolor{codepurple}{rgb}{0.58,0,0.82}
\definecolor{backcolour}{rgb}{0.95,0.95,0.92}

\lstdefinestyle{mystyle}{
    backgroundcolor=\color{backcolour},
    commentstyle=\color{codegreen},
    keywordstyle=\color{magenta},
    numberstyle=\tiny\color{codegray},
    stringstyle=\color{codepurple},
    basicstyle=\ttfamily\footnotesize,
    breakatwhitespace=false,
    breaklines=true,
    captionpos=b,
    keepspaces=true,
    numbers=left,
    numbersep=5pt,
    showspaces=false,
    showstringspaces=false,
    showtabs=false,
    tabsize=2
}

\lstset{style=mystyle}

\newtheorem{theorem}{Theorem}[section]
\theoremstyle{definition}
\newtheorem{problem}{Problem}[section]
\renewcommand\qedsymbol{QED}


\title{Functional Programming \\ Using Haskell}

\author{Bruno Giao}

\date{December 1, 2022}

\begin{document}

\maketitle

\section{Exercises}
Let us consider the following functions:
\begin{itemize}
    \item{length} (size),
    \item{(++)} (merge),
    \item{reverse} (reverse),
    \item{nub} (removal of repeats),
    \item{words} (words of a phrase),
    \item{unwords} (inverse of previous),
    \item{sort} (sort),
    \item{(==)} (equality),
    \item{lines} (lines of a text),
    \item{unlines} (inverse of previous),
    \item{take} (get prefix),
    \item{drop} (get sufix).
\end{itemize}
Now run these commands in GHCi:
\begin{align*}
    &\text{:m Data.Ratio}\\
    &\text{:m Data.Char}\\
    &\text{:m Data.List}
\end{align*}
\subsection*{Words}
Solve these problems by passing these words as inputs
to the correct functions in GHCi. (Do not forget the `` '')
\begin{problem}
    How many letters does ``Rainbow'' contain?
\end{problem}
\begin{problem}
    How can we merge ``Has'' with ``kell''?
\end{problem}
\begin{problem}
    How can we reverse ``anilina'' and ``direito''?
\end{problem}

\begin{problem}
    Run the following instructions:
    \begin{align*}
        &\text{nub ``direito''}\\
        &\text{nub ``anilina''}
    \end{align*}
    What do you think ``nub'' does?
\end{problem}

\subsection*{Sentences}
Run this command:
\[\text{sentence = ``Jean Michel Powerly Sawyer''}\]
\begin{problem}
    What do these intructions do?
    \begin{align*}
        &\text{words sentence}\\
        &\text{unwords (words sentence)}
    \end{align*}
\end{problem}

\begin{problem}
    What's the difference between these instructions?
    \begin{align*}
        &\text{words (reverse sentence)}\\
        &\text{reverse (words sentence)}
    \end{align*}
\end{problem}

\begin{problem}
    What happens when we run the following instruction? What can we conclude?
    \begin{align*}
        \text{words (unwords sentence)}
    \end{align*}
\end{problem}

\begin{problem}
    Call back to Problem 1.5. \hfill \\
    What if the input was ``\hspace{1cm}Help\hspace{2cm}Drowned in\hspace{1cm} Spaces''
    \hfill\\This is a very important effect if we want to re-use it, we use:
    \[\text{effect x = unwords (words x)}\]
    Run effect with that input and with sentence.
\end{problem}
\begin{problem}
    Run these commands:
    \begin{align*}
        &\text{take 1 (words sentence)}\\
        &\text{take 2 (words sentence)}\\
        &\text{take 3 (words sentence)}
    \end{align*}
    What does take do?
\end{problem}
\begin{problem}
    Run these commands:
    \begin{align*}
        &\text{drop 1 (words sentence)}\\
        &\text{drop 2 (words sentence)}\\
        &\text{drop 3 (words sentence)}
    \end{align*}
    What does drop do?
\end{problem}
\begin{problem}
    Consider the following commands:
    \begin{align*}
        &\text{unwords (take 2 (words sentence)) }\\
        &\text{unwords (drop 2 (words sentence)) }
    \end{align*}
    How can we make these commands work for any sentence?\hfill\\
    Try your solution for any sentence.
\end{problem}
\subsection*{Texts}
Run this command:
\[\text{text = [``Hello I am studying'',``No you are not'',``I am from Ancient Greece'']}\]
\begin{problem}
What do you think this instruction does? Run it and see if it matches your expectations?
\[\text{sort text}\]
\end{problem}
\begin{problem}
    How can we count the number of phrases using haskell?
\end{problem}
\begin{problem}
    Run:
    \[\text{unlines text}\]
    Interpret the result.\hfill\\
    What about the command:
    \[\text{lines (unlines text)}\]
\end{problem}
\newpage

\section{Summary}
Using and interpreting Pre-Defined functions in the GHC haskell interpreter

\end{document}
